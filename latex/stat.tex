
\documentclass[12pt]{article}

\usepackage[margin=1in]{geometry}
\usepackage{amsmath,amsthm,amssymb,scrextend}
\usepackage{color}
\usepackage{fancyhdr}
\usepackage{graphicx}
\usepackage{lineno}
\usepackage{listings}
\usepackage{url}
\usepackage{array}

\pagenumbering{gobble}
\newcommand*{\C}[2]{{}^{#1}C_{#2}}%

\begin{document}
\section{stuff}
\subsection{central tendenceis}
{\large mean ($\mu$):}
\begin{itemize}
  \item direct: $\displaystyle \frac{\sum{f_ix_i}}{n}$
  \item assumed: $\displaystyle a + \frac{\sum{f_id_i}}{n} \\ d_i = x_i - a $
  \item step deviation: $\displaystyle a + \frac{\sum{f_iu_i}}{n}h \\ u_i = \frac{x_i - a}{h} $
\end{itemize}
\vspace*{1\baselineskip}
{\large median: \\ } \\
$\displaystyle l + \frac{h}{f}(\frac{n}{2}-cf) $ \\
l = lower limit of median class \\
h = width of median class \\
f = frequency of median class \\
n = $\sum{f}$ \\
cf = cumulative frequency of pre-median class \\
median class = class with highest frequency \\
\vspace*{1\baselineskip} \\
{\large mode: \\ } \\
$\displaystyle l + \frac{f_1 - f_0}{2f_1 - f_0 - f_2}h $ \\
l = lower limit of modal class \\
h = width of modal class \\
$f_0$ = frequency of pre-modal class \\
$f_1$ = frequency of modal class \\
$f_2$ = frequency of post-modal class \\
\vspace*{1\baselineskip} \\
{\large standard deviation: \\ } \\
$\displaystyle \frac{\sum{(x_i - \mu)^2}}{n}$
\vspace*{1\baselineskip} \\

\subsection{probablity}
$P(A \cap B) = P(A|B) P(B) = P(B|A) P(A)$

\subsection{mean}
AM = $\displaystyle \frac{(x_1 + x_2 + ... + x_n)}{n}$ \\ \\
GM = $\displaystyle {}^n\sqrt{x_1\ x_2\ ...\ x_n}$ \\ \\
HM = $\displaystyle \frac{n}{(1/x_1)+(2/x_2)+...+(1/x_n)}$ \\ \\ \\
$GM^2 = AM \times HM$

\subsection{integration}
$\displaystyle
\int{x^ndx} = \frac{x^{n+1}}{n+1} \\
\int{e^ndx} = e^n \\
\int{a^xdx} = \frac{a^x}{\log{a}} \\
\int{\frac{1}{x}dx} = \log{x} \\
\int{uvdx} = u\int{vdx} - \int{(\frac{d}{dx}u \int{vdx})dx} \\
$
\newpage

\section{discrete}
$\displaystyle
mean (\mu)
= E(x)
= \sum_{i=1}^n{f_ix_i} \\
variance (\sigma^2)
= Var(x)
= E(x^2) - \mu^2
= \sum_{i=1}^n{f_ix_i^2} - (\sum_{i=1}^n{f_ix_i})^2  \\
standard deviation (\sigma)
= \sqrt{Var(x)}
$

\subsection{binomial (bernauli)}
\begin{itemize}
  \item n is small
  \item two outcomes: success (p) or fail (q)
\end{itemize}
$
pmf = f(X=x) = \C{n}{x} p^x q^{n-x} \\ \\
mean (\mu) = np \\
variance (\sigma^2) = npq \\
standard deviation (\sigma) = \sqrt{npq}
$

\subsection{poisson}
\begin{itemize}
  \item n is large
  \item p is very small
\end{itemize}
$
pmf = f(X=x, \lambda = np) = \displaystyle\frac{e^{-\lambda}\lambda^x}{x!} \\ \\
mean (\mu) = \lambda = np \\
variance (\sigma^2) = \lambda = np \\
standard deviation (\sigma) = \sqrt\lambda = \sqrt{np}
$
\newpage

\section{continuous}
\subsection{uniform}
$\displaystyle
pdf = f(\alpha < x < \beta) = \frac{1}{\beta - \alpha} \\ \\
mean (\mu) = E(x) = \frac{\alpha + \beta}{2} \\
variance (\sigma^2) = Var(x) = \frac{(\beta - \alpha)^2}{12} \\
standard deviation (\sigma) =\frac{\beta - \alpha}{\sqrt{12}}
$

\subsection{exponential}
$\displaystyle
pdf = f(0 \leq x < \infty) = \lambda e^{-\lambda x} \\
pdf = f(0 \leq x < a) = 1 - e^{-\lambda a} \\ \\
mean (\mu) = E(x) = \frac{1}{\lambda} \\
variance (\sigma^2) = Var(x) = \frac{1}{\lambda^2} \\
standard deviation (\sigma) = \sqrt\lambda = \sqrt{np}
$

\subsection{normal}
$\displaystyle
pdf = f(-\infty < x < \infty, \mu, \sigma) = \frac{1}{\sigma\sqrt{2\pi}} e^{-\frac{1}{2}(\frac{x-\mu}{\sigma})^2} \\ \\
mean (\mu) = E(x) = \int_{-\infty}^{\infty}{xf(x)dx} \\
variance (\sigma^2) = Var(x) = \int_{-\infty}^{\infty}{(x-\mu)^2f(x)dx} \\
standard deviation (\sigma) = \sqrt\lambda = \sqrt{np}
$

\subsection{log normal}
\subsection{standard normal}
$\displaystyle
pdf = f(\frac{x-\mu_1}{\sigma} < z < \frac{x-\mu_2}{\sigma}) = f(z < \frac{x-\mu_2}{\sigma}) - f(z < \frac{x-\mu_1}{\sigma})\\ \\
\mu = n \pm m
$

\subsection{gamma}
$\displaystyle
pdf = f(X=x, \lambda, r) = \frac{\lambda^r x^{r-1} e ^{-\lambda x}}{\Gamma r} \\
\Gamma r = \int_{0}^{\infty}{x^{r-1}e^{-x}dx} \\
0 < x,r \\ \\ \\
mean (\mu) = E(x) = \frac{r}{\lambda} \\
variance (\sigma^2) = Var(x) = \frac{r}{\lambda^2} \\
$

\section{joint distribution}
$\displaystyle P(a<x<b, c<y<d) = \int_a^b{\int_c^d{f(x,y) dx}}$

\section{skewness}
$\displaystyle \frac{(mean - mode)}{std\ dev}$
\subsection{Karl Pearson's measure of skewness}
$\displaystyle \frac{3(mean - median)}{std\ dev}$

\subsection{Bowley's measure of skewness}
$\displaystyle \frac{Q_3 - 2 median + Q_1}{Q_3 - Q_1}$ \\ \\
$Q_1$ = find median with median class as $N/4$ \\
$Q_2$ = find median with median class as $N/2$ \\
$Q_2$ = find median with median class as $3N/4$

\subsection{Moment's measure of skewness}
$\displaystyle \mu_r = \frac{1}{n} \sum{f_i(x_i -  \bar{x})^r}$ \\ \\
$\displaystyle \mu_r = \frac{h^r}{n} \sum{f_i\ x_i^r}$\ \ \ (for assumed mean) \\ \\
$\displaystyle \beta_1 = \frac{{\mu_3}^2}{{\mu_2}^3}$ \\ \\
$\displaystyle \gamma_1 = \pm \sqrt{\beta_1} = \frac{\mu_3}{\sigma^3} = \alpha_3$
\newpage

\subsection{kurtosis}
$\displaystyle \beta_2 = \frac{\mu_4}{{\mu_2}^2}$ \\ \\
$\displaystyle \gamma_2 = \beta_2 - 3$ \\ \\
$\beta_2 = 3$, meso kurtic \\
$\beta_2 < 3$, platy kurtic \\
$\beta_2 > 3$, lepto kurtic

\section{correlation}
\subsection{Karl Pearson's coefficient of correlation}
$\displaystyle r = \frac{ n \sum{xy} - \sum{x}\sum{y} }{
\sqrt{n\sum{x^2} - {(\sum{x})}^2 } \sqrt{n\sum{y^2} - {(\sum{y})}^2 } }$ \\ \\ \\
$\displaystyle r = \frac{Cov(x, y)}{\sigma_x\ \sigma_y}$

\subsection{Spearmann's rank correlation}
$\displaystyle \rho = 1 - \frac{6\sum{d^2}}{n(n^2 - 1)} \\ \\
d_i = Rx_i - Ry_i $
\subsection{concurrent deviations}
$\displaystyle rc = \sqrt{\frac{2C-N}{N}} \\ \\
C: +ve\ \ \delta_x \delta_y$

% \vspace*{6\baselineskip}
\end{document}
