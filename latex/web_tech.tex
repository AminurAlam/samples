\documentclass[12pt]{article}

\usepackage[margin=1in]{geometry}
\usepackage{./extracommands}
\usepackage{./extraoptions}
\usepackage{amsmath,amsthm,amssymb,scrextend}
\usepackage{color}
\usepackage{fancyhdr}
\usepackage{graphicx}
\usepackage{lineno}
\usepackage{listings}
\usepackage{url}
\usepackage{array}

\begin{document}
\generatetitlepage{WEB TECHNOLOGY \\ LAB ASSIGNMENT}{Aminur Alam}{roll}{$n^{th}$}{dept}{course}{institution}

% index
\Huge
\newgeometry{left=2cm,right=2cm,top=1in,bottom=1in}
\centering \underline{INDEX} \\
\vspace*{3\baselineskip}
\setlength{\arrayrulewidth}{0.4mm}
\renewcommand{\arraystretch}{1.8}
\setlength{\tabcolsep}{10pt}

\LARGE
\begin{tabular}{|m{8.5cm}|m{2.2cm}|m{3cm}|}
  \hline
  PROJECT & DATE & SIGN \\
  \hline
  \hline
  1. HTML tags overview & 06 Feb & \\
  2. Date and time & 20 Feb & \\
  3. Turning a bulb on and off & 20 Feb & \\
  4. Facebook login page & 27 Feb & \\
  5. Form validation & 06 Mar & \\
  6. Frame animation & 13 Mar & \\
  7. Chatbox & 20 Mar & \\
  8. Scroll back button & 17 Apr & \\
  9. Showcase popup boxes & 24 Apr & \\
  10. Responsive webpage & 24 Apr & \\
  11. Flashcard UI & 1 MAy & \\
  \hline
\end{tabular}
\newpage
\restoregeometry
\normalsize

\section{HTML tags overview}
\codequest{}{html/tags/index.html}
\shot{tags.png}{8cm}
\newpage

\section{Date and Time}
\codequest{}{html/datetime/index.html}
\shot{datetime_before.png}{6.8cm}
\shot{datetime_after.png}{6.8cm}

\section{Turning a bulb on and off}
\codequest{}{html/bulb/index.html}
\shot{bulb_off.png}{7cm}
\shot{bulb_on.png}{7cm}
\newpage

\section{Facebook login page}
\codequest{}{html/fb/index.html}
\shot{fb_login.png}{7.4cm}
\newpage

\section{Facebook form validation}
\codequest{}{html/fb/signup.html}
\shot{fb_signup.png}{9cm}

\section{Frame animation}
\codequest{}{html/animation/index.html}
\newpage
\fbox{\includegraphics[totalheight=4.4cm]{~/repos/samples/html/animation/bounce/1.jpg}}
\fbox{\includegraphics[totalheight=4.4cm]{~/repos/samples/html/animation/bounce/12.jpg}}
\fbox{\includegraphics[totalheight=4.4cm]{~/repos/samples/html/animation/bounce/21.jpg}}
\fbox{\includegraphics[totalheight=4.4cm]{~/repos/samples/html/animation/bounce/30.jpg}}
\fbox{\includegraphics[totalheight=4.4cm]{~/repos/samples/html/animation/bounce/40.jpg}}
\fbox{\includegraphics[totalheight=4.4cm]{~/repos/samples/html/animation/bounce/40.jpg}}
\fbox{\includegraphics[totalheight=4.4cm]{~/repos/samples/html/animation/bounce/70.jpg}}
\fbox{\includegraphics[totalheight=4.4cm]{~/repos/samples/html/animation/bounce/100.jpg}}
\newpage

\section{Chatbox}
\vspace*{1\baselineskip}
\codequest{}{html/chat/index.html}
\vspace*{1\baselineskip}
\codequest{}{html/chat/script.js}
\shot{chat.png}{7.21cm}
\newpage

\section{Scrollback button}
\codequest{}{html/scroll/index.html}
\shot{scroll_bot.png}{9cm}
\shot{scroll_top.png}{9cm}
\newpage

\section{Showcase popup boxes}
\codequest{}{html/popup/index.html}
\vspace*{3\baselineskip}
\shot{popup_result.png}{7cm}
\shot{popup_alert.png}{7.2cm}
\shot{popup_confirm.png}{7.2cm}
\shot{popup_prompt.png}{7.2cm}
\newpage

\section{Responsive webpage}
\codequest{}{html/size/index.html}
\shot{size_full.png}{9.1cm}
\shot{size_half.png}{9.1cm}
\newpage

\section{Flashcard UI}
{\Large \raggedright
  Anki is a flashcard program used to help you remember things. \\
  It also lets you build your own UI using tools like handlebars.js and ruby text. \\
}
\vspace*{3\baselineskip}

% FRONT
\codequest{}{~/repos/dotfiles/anki/front.html}
\shot{anki_front.png}{9.1cm}
{\Large \raggedright
  This is the front side of the flashcard, here haldelbars is used to fetch the value of \verb|Word| and \verb|Sentence| we would like to remember. \\
}
\newpage

% BACK
\codequest{}{~/repos/dotfiles/anki/back.html}
\shot{anki_back.png}{9.1cm}
{\Large \raggedright
  This is the back side of the flashcard, in addition to \verb|Word| and \verb|Sentence| we now also render their meanings. \\
  \vspace*{1\baselineskip}
  The \verb|furigana:| prefix is used in handlebars to render pronunciation as ruby text above, and the \verb|{{#foo}}...{{/foo}}| syntax tells handlebars to only render the content inside if \verb|foo| is defined. \\
  \vspace*{1\baselineskip}
  \verb|<details>| and \verb|<summary>| are used to create a dropdown that hide additional metadata that can be important but can also be annoying if its always visible. Clicking on the \verb|+| button reveals this data. \\
  \vspace*{1\baselineskip}
  The javascript here is used to make numbers more human readable (\verb|28531 -> 28k|) \\
}
\newpage

% STYLE
\codequest{}{~/repos/dotfiles/anki/style.css}
\vspace*{1\baselineskip}
{\Large \raggedright
  This stylesheet uses tricks like setting custom font only for certain unicode ranges and responsive UI depending on orientation \\
}
\end{document}
